% !TEX program = xelatex
% This is my resume
% Chinese translation
% by Hexi

\documentclass{resume}

\usepackage{lastpage}
\usepackage{fancyhdr}
\usepackage{linespacing_fix} % disable extra space before next section
\usepackage[fallback]{xeCJK}

%% \setmainfont[]{SimSun}
%% \setCJKfallbackfamilyfont{rm}{HAN NOM B}
%% \setCJKmainfont[BoldFont=Sarasa Gothic SC Regular,ItalicFont=Sarasa Gothic SC Italic]{Source Han Serif SC}
%% \renewcommand{\thepage}{\Chinese{page}}

\begin{document}
% \pagestyle{fancy}
% \fancyhf{}
\renewcommand\headrulewidth{0pt}
% \cfoot{\thepage\ of \pageref{LastPage}}

\name{李晨曦}

\basicInfo{
  \email{i@hexilee.me} \textperiodcentered\ 
  \phone{(+86) 18868104342} \textperiodcentered\ 
  \github[Hexilee]{https://github.com/hexilee}
  % \linkedin[user]{https://www.linkedin.com/in/user}
}

\section{\texorpdfstring{\faGraduationCap}{}\ 教育经历}
\datedsubsection{\textbf{浙江大学}, 中国}{2016.9 -- 现在}
  专业:海洋工程与技术,预计毕业日期:2021.6

\section{\texorpdfstring{\faUsers}{}\ 工作经历}
\textbf{暂无}

\section{\texorpdfstring{\faGithubAlt}{}\ 个人项目}

\datedsubsection{\textbf{\href{https://github.com/Hexilee/roa}{roa}}}{}
给rust实现的类koajs异步web框架。
\begin{itemize}
  \item 实现了一个基于hyper,近乎零成本抽象的roa-core,支持自定义的异步运行时和传输层协议。
  \item 实现了分别基于tokio和async-std的运行方案;支持TLS。
  \item 实现了基于Radix Trie和正则表达式的路由。
  \item 实现了很多常用中间件和Context功能扩展(如Content Compression, Query Parser, CORS, Cookie, JWT, WebSocket)。
\end{itemize}

\datedsubsection{\textbf{\href{https://github.com/Hexilee/async-io-demo}{async-io-demo}}}{}
阐述了个人对rust异步编程的理解,基于mio给rust实现了试验性的异步运行时及其它IO组件。
\begin{itemize}
  \item 介绍了几种常见的异步API实现并分析优劣。
  \item 分析rust在实现stackless coroutine时遇到的问题和现有的解决方案。
  \item 实现了异步的TcpStream和fs::\{read, write\}。
  \item 使用实现出的TcpStream和fs写了几个example验证实现的正确性。
\end{itemize}

\datedsubsection{\textbf{\href{https://github.com/Hexilee/tio}{tio}}}{}
一套rust异步IO解决方案,目前只完成了task和net两个模块。
\begin{itemize}
  \item task模块提供带Work Stealing的多线程异步运行时、简单同步任务运行时和异步sleep,timer等。
  \item net模块提供TcpListener, TcpStream, UdpSocket, UnixListener, UnixStream和UnixDatagram。
\end{itemize}

\datedsubsection{\textbf{\href{https://github.com/Hexilee/unhtml.rs}{unhtml.rs}}}{}
给rust实现的html parser,基于proc-macro和scraper。
\begin{itemize}
  \item 支持rust所有的非引用类型、无泛型参数类型、Vec和Option。
  \item 使用css-selector,支持缺省值。
\end{itemize}


\section{\texorpdfstring{\faCogs}{}\ 技能}
\begin{itemize}[parsep=0.25ex]
  \item \textbf{编程语言}:
    \textbf{泛语言开发者}(编程不受特定语言限制),
    且尤其熟悉Rust/Cpp/Go/NodeJS/TS/Python/Java,
    较为熟悉C/Kotlin/Elixir/Scala。

  \item \textbf{Web后端}:
    使用过各种语言和框架(包括django, koa, gin, echo, beego, springboot, nest, vertx, phoenix)写过Web后端,对不同类型后端框架的实现很感兴趣也有一定研究,开发并维护过拥有上千用户的后端项目。

  \item \textbf{HTTP协议}:
    比较熟悉,读过RFC-7230和RFC-7231,并翻译了一部分。
  
  \item \textbf{异步编程}:
    对多路复用IO(如epoll, kqueue)、异步IO(如aio, io-uring, iocp)均有一定研究。对stackful/stackless coroutine的应用场景、优劣、实现原理均有研究。

  \item \textbf{并发模型效率及并发安全}:  
    对传统多线程master-worker、CSP、Actor都比较熟悉;对锁(如读写锁和Linux上的Futex),Atomic(Memory Order),CAS的实现和效率也有所了解。

  \item \textbf{Linux部署/运维}:
    部署项目主要使用pm2和docker(与gitlab集成CD)。有过两年的线上生产运维经验,能熟练查看机器资源占用情况、进行基本的网络诊断、使用iptables转发网络包以及使用ssh打tunnel等。

  \item \textbf{开发工具/习惯}:
    能适应任何编译器/操作系统,平时在macOS下使用JetBrains IDE、vim、vscode。熟练使用GitHub,GitLab,travis-ci,coveralls,gitlab-ci等进行团队协作和代码测试。喜欢使用静态类型语言和编写测试来降低自己的心智负担,喜欢写完善的文档来向他人介绍展示自己的项目。
\end{itemize}

\section{\texorpdfstring{\faInfo}{}\ 其他}
\begin{itemize}[parsep=0.25ex]
  \item 博客: \url{https://hexilee.me/}。
  \item 知乎: \url{https://www.zhihu.com/people/hexilee}。
  \item 岗位相关:曾参加过TiDB Hackathon 2019,实现了prof结果的HTTP API;今年五月报名了易用性挑战赛,目前解决了\href{https://github.com/tikv/tikv/issues/7626}{\#7626};对Raft算法、TiKV的架构均有所了解。
\end{itemize}

\end{document}
