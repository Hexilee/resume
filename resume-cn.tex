% !TEX program = xelatex
% This is my resume
% Chinese translation
% by Hexi

\documentclass{resume}

\usepackage{lastpage}
\usepackage{fancyhdr}
\usepackage{linespacing_fix} % disable extra space before next section
\usepackage[fallback]{xeCJK}

%% \setmainfont[]{SimSun}
%% \setCJKfallbackfamilyfont{rm}{HAN NOM B}
%% \setCJKmainfont[BoldFont=Sarasa Gothic SC Regular,ItalicFont=Sarasa Gothic SC Italic]{Source Han Serif SC}
%% \renewcommand{\thepage}{\Chinese{page}}

\begin{document}
% \pagestyle{fancy}
% \fancyhf{}
\renewcommand\headrulewidth{0pt}
% \cfoot{\thepage\ of \pageref{LastPage}}

\name{李晨曦}

\basicInfo{
  \email{hexileee@gmail.com} \textperiodcentered\ 
  \phone{(+86) 18868104342} \textperiodcentered\ 
  \github[Hexilee]{https://github.com/hexilee}
  % \linkedin[user]{https://www.linkedin.com/in/user}
}

\section{\texorpdfstring{\faGraduationCap}{}\ 教育经历}
\datedsubsection{\textbf{浙江大学}, 中国}{2016.9 -- 现在}
  专业:海洋工程与技术,预计毕业日期:2020.6

\section{\texorpdfstring{\faUsers}{}\ 工作经历}
\textbf{暂无}

\section{\texorpdfstring{\faGithubAlt}{}\ 个人项目}

\datedsubsection{\textbf{\href{https://github.com/Hexilee/async-io-demo}{async-io-demo}}}{}
阐述了个人对 rust 异步 IO 的理解,并基于\href{https://github.com/carllerche/mio} {mio} 和 coroutine 给 rust 实现了试验性的异步非阻塞 IO。
\begin{itemize}
  \item 介绍了几种可能的异步 API 实现并分析优劣。
  \item 分析 rust 在实现 stackless coroutine 时遇到的问题和现有的解决方案。
  \item 分别基于 rust-1.32 和 rust-1.33 实现了异步非阻塞的 TcpStream 和 fs::{read, write}。
  \item 使用实现出的 TcpStream 和 fs 写了几个 example 验证实现的正确性。
\end{itemize}

\datedsubsection{\textbf{\href{https://github.com/Hexilee/unhtml.rs}{unthml.rs}}}{}
给 rust 实现的 html parser,基于 proc-macro 和 \href{https://github.com/programble/scraper} {scraper}。
\begin{itemize}
  \item 支持 rust 所有的非引用、无泛型参数类型和 Vec。
  \item 使用 css-selector,支持缺省值。
\end{itemize}

\datedsubsection{\textbf{\href{https://github.com/Hexilee/unhtml}{unthml}}}{}
unhtml.rs 的 go 版本,使用 struct tag 作标注,基于 \href{github.com/PuerkitoBio/goquery} {goquery}。
\begin{itemize}
  \item 支持所有的 go 类型,支持插入 coverter。 
  \item 使用 css-selector,缺省值使用零值。
\end{itemize}

\datedsubsection{\textbf{\href{https://github.com/Hexilee/htest}{htest}}}{}
 给 go 写的 http 测试库,API 受 Spring Framework 的启发。
\begin{itemize}
  \item 支持 go 标准库中 http 包所支持的各种 http 请求。
  \item 对 JSON、XML 等常见数据格式的测试有便捷的 API 支持。对媒体数据的测试有便捷的 hash 比对支持。
\end{itemize}

\datedsubsection{\textbf{\href{https://github.com/rady-io/inject}{inject}}}{}
给 go 写的基于 struct tag 的依赖注入框架,类似于 Spring Boot。由于 go 语言本身的设计限制,现已弃坑;框架设想见 \href{https://zhuanlan.zhihu.com/p/32616898} {给 Go 写一个类 Spring Boot 框架}。截至最后一次 release(v0.4.3),该项目已实现了设想文章中的全部功能,包括:
\begin{itemize}
  \item 全局组件单例注入,局部作用域单例注入,配置文件注入。
  \item 结构化的路由注册(router、controller 和 middleware 可以嵌入其它的 router)。
  \item 中间件注册。
  \item 组件自定义构造器。
  \item Entities 的注册。
  \item 配置文件热重载。
  \item 一些 echo middleware 的 wrapper(cors、jwt、logger)。
  \item 依赖注入测试。
  \item 环境变量控制加载配置文件。
\end{itemize}


\section{\texorpdfstring{\faCogs}{}\ 技能}
\begin{itemize}[parsep=0.25ex]
  \item \textbf{编程语言}:
    \textbf{泛语言开发者}(编程不受特定语言限制),
    且尤其熟悉 Rust/Cpp/Go/NodeJS/TS/Python/Java,
    较为熟悉 C/Kotlin/Dart/Elixir/Scala (均不分先后)。

  \item \textbf{Web 后端}:
    使用过各种语言和框架(包括 django, koa, gin, echo, beego, springboot, nest, vertx, phoenix)写过 Web 后端,对不同类型后端框架的实现很感兴趣也有一定研究,开发并维护过拥有上千用户的后端项目。除了传统 HTTP 后端,也开发过 Websocket、gRPC 接口的后端。

  \item \textbf{HTTP 协议}:
    比较熟悉,读过 \href{https://tools.ietf.org/html/rfc7230} {RFC-7230} 和 \href{https://tools.ietf.org/html/rfc7231} {RFC-7231},并翻译了\href{https://github.com/Hexilee/RFC-HTTP_1.1-zh} {一部分}。

  \item \textbf{并发模型效率及并发安全}:  
    对传统多线程 master-worker、CSP、Actor 都比较熟悉;对 stackful coroutine 和 stackless coroutine 的实现原理均有研究。对锁(主要是 Linux 上的 Futex),Atomic(Memory Order),CAS 的优劣及效率也有所了解。

  \item \textbf{Linux 部署/运维}:
    部署项目主要使用 pm2 或 docker(与 gitlab 集成 CD)。在浙江大学求是潮技术研发中心有过两年的线上生产运维经验,能熟练查看机器资源占用情况、进行基本的网络诊断、使用 iptables 转发网络包以及使用 ssh 打 tunnel 等。
  
  \item \textbf{Web 前端}:
    对现代前端框架感兴趣并有一定了解,主要熟悉 React-Redux、单向数据流。由于很长时间没写过实际的前端项目,对具体布局和浏览器兼容性等内容并不了解。

  \item \textbf{开发工具/习惯}:
    能适应任何编译器/操作系统,平时在 macOS 下使用 JetBrains IDE、vim、vscode。熟练使用 GitHub,GitLab,travis-ci,coveralls,gitlab-ci 等进行团队协作和代码测试。喜欢使用静态类型语言和编写测试来降低自己的心智负担,喜欢写完善的文档来向他人介绍展示自己的项目。
\end{itemize}

\section{\texorpdfstring{\faInfo}{}\ 其他}
\begin{itemize}[parsep=0.25ex]
  \item 博客: \url{https://hexilee.me/}。
\end{itemize}

\end{document}
